\documentclass{article}
\usepackage{hyperref} % Pour les liens hypertexte

\begin{document}

\title{Analyse de code avec Sonar : Problèmes et Solutions}
\author{MOHAMED KHABOU}
\date{\today}

\maketitle

\section{Introduction}

Ce document présente une analyse du code source en C++ à l'aide de Sonar, en mettant en évidence les problèmes liés aux règles de code identifiés dans le code source. Les problèmes identifiés sont suivis des solutions recommandées pour les résoudre.

\section{Problèmes de Code}

\subsection{'logo' has virtual functions but non-virtual destructor (cpp:S1235)}

Le problème ici est que la classe `logo` a des fonctions virtuelles, mais elle n'a pas de destructeur virtuel. Cela peut entraîner des fuites de mémoire si des objets de classes dérivées sont supprimés via un pointeur vers la classe de base.

Solution recommandée : Ajouter un destructeur virtuel à la classe "logo".

\subsection{'logoCercle' has virtual functions but non-virtual destructor (cpp:S1235)}

Le même problème que précédemment s'applique également à la classe `logoCercle`.

Solution recommandée : Ajouter un destructeur virtuel à la classe `logoCercle`.

\subsection{'logoRectangle' has virtual functions but non-virtual destructor (cpp:S1235)}

Le même problème que précédemment s'applique également à la classe `logoRectangle`.

Solution recommandée : Ajouter un destructeur virtuel à la classe `logoRectangle`.

\subsection{Make sure that using this pseudorandom number generator "rand" is safe here (cpp:S2245)}

La règle Sonar recommande de s'assurer que l'utilisation de la fonction `rand` est sécurisée. Il existe des problèmes potentiels de portabilité et de qualité des nombres aléatoires avec `rand`.

Solution recommandée : Remplacer l'utilisation de `rand` par les installations fournies dans la bibliothèque `<random>`.

\subsection{Replace "new" with an operation that automatically manages the memory (cpp:S5025)}

Sonar recommande de remplacer l'utilisation de `new` par des opérations de gestion automatique de la mémoire pour éviter les fuites de mémoire potentielles.

Solution recommandée : Utiliser des classes de gestion de la mémoire comme std unique ptr  ou std shared ptr  pour gérer automatiquement la mémoire des objets.

\section{Conclusion}

Ce document a présenté les problèmes identifiés dans le code source en utilisant les règles de code Sonar, ainsi que les solutions recommandées pour les résoudre. Il est important de suivre ces recommandations pour améliorer la qualité et la sécurité du code.

\end{document}
